\documentclass[
paper=a4,                       % paper size
fontsize=11pt,                  % font size
headinclude=false,              % header does not belong to the text
footinclude=false,              % footer does not belong to the text
pagesize,                       % set the pagesize in a DVI document
]{scrartcl}

% Copyright (C) 2010,2011,2012,2013,2014,2015,2016 The ESPResSo project
% Copyright (C) 2002,2003,2004,2005,2006,2007,2008,2009,2010
%  Max-Planck-Institute for Polymer Research, Theory Group
%  
% This file is part of ESPResSo.
%   
% ESPResSo is free software: you can redistribute it and/or modify it
% under the terms of the GNU General Public License as published by the
% Free Software Foundation, either version 3 of the License, or (at your
% option) any later version.
%  
% ESPResSo is distributed in the hope that it will be useful, but
% WITHOUT ANY WARRANTY; without even the implied warranty of
% MERCHANTABILITY or FITNESS FOR A PARTICULAR PURPOSE.  See the GNU
% General Public License for more details.
%  
% You should have received a copy of the GNU General Public License
% along with this program.  If not, see <http://www.gnu.org/licenses/>.
%
\usepackage[draft]{varioref}    % defines \vref
\usepackage{hyperref}           % automatically creates links when
                                % using pdflatex, defines \url
\usepackage{ifpdf}              % defines \ifpdf
\usepackage{graphicx}           % handles graphics
\usepackage{color}              % use colors

\usepackage{amsmath}

\usepackage{verbatim}           % required for \verbatim and \endverbatim
\usepackage{fancyvrb}
\usepackage{calc}               % compute length
\usepackage{ifthen}             % provide ifthen
\usepackage{xspace}
\usepackage{units}
\usepackage[numbers]{natbib}

\usepackage{listings}

% For building the distribution docs, disable todo boxes.
%\usepackage[disable]{todonotes}
\usepackage{todonotes}

\newcommand{\es}{\mbox{\textsf{ESPResSo}}\xspace}
\newcommand{\ie}{\textit{i.e.}\xspace}
\newcommand{\eg}{\textit{e.g.}\xspace}
\newcommand{\etal}{\textit{et al.}\xspace}

\newcommand{\codebox}[1]%
{\texttt{#1}}

\DefineVerbatimEnvironment{code}{Verbatim}%
{commandchars=\\\{\}}
\makeatletter
\newenvironment{tclcode}
{%
  \addtolength{\linewidth}{-2em}% set the line length
  \@minipagetrue%%%DPC%%%
  \@tempswatrue%%%DPC%%%
  \hsize=\linewidth%
  \setbox0=\vbox\bgroup\verbatim
}{\endverbatim
  \unskip\setbox0=\lastbox%%%DPC%%%
  \egroup
  \par%
  \noindent\hspace{1em}%
  \codebox{\box0}%
  \par\noindent%
}
\makeatother

% \newcommand{\todo}[1]{
%   \marginpar{%
%     \setlength{\fboxrule}{1pt}
%     \fcolorbox{red}{yellow}{%
%       \parbox{\marginparwidth-2\fboxrule-2\fboxsep}{%
%         \bf\raggedright\scriptsize #1%
%       }%
%     }%
%   }%
% }

\makeatletter
\renewcommand{\minisec}[1]{\@afterindentfalse \vskip 1.5ex
  {\parindent \z@
    \raggedsection\normalfont\sffamily\itshape\nobreak#1\par\nobreak}%
  \@afterheading}
\makeatother

\newcommand{\esptitlehead}{
  \titlehead{
    \begin{center}
      \includegraphics[width=5cm]{logo/logo.pdf}
    \end{center}
  }
}

% Do magic that \bfseries for keywords actually works
\DeclareFontShape{OT1}{cmtt}{bx}{n}{<5><6><7><8><9><10><10.95><12><14.4><17.28><20.74><24.88>cmttb10}{}

% Custom colors
\definecolor{stringblue}{rgb}{0.09,0.211,0.57}
\definecolor{codered}{rgb}{0.655,0.1137,0.3747}
\definecolor{deepgreen}{rgb}{0,0.5,0}
\definecolor{commentgray}{rgb}{0.4,0.4,0.4}
\definecolor{framegray}{rgb}{0.8,0.8,0.8}

% Python style for highlighting
\newcommand\pythonstyle{\lstset{
    language=Python,
    belowskip=\bigskipamount,
    aboveskip=\bigskipamount,
    basicstyle=\ttfamily\small,
    otherkeywords={self},             % Add keywords here
    keywordstyle=\bfseries\color{codered},
    emph={MyClass,__init__},          % Custom highlighting
    emphstyle=\bfseries\color{deepgreen},    % Custom highlighting style
    commentstyle=\color{commentgray}\ttfamily,
    stringstyle=\color{stringblue},
    frame=tb,                         % Any extra options here
    rulecolor=\color{framegray},
    showstringspaces=false            % 
    }
}


% Python environment
\lstnewenvironment{python}[1][]{\pythonstyle \lstset{#1}}{}

% Python for external files
\newcommand\pythonexternal[2][]{{\pythonstyle \lstinputlisting[#1]{#2}}}

% Python for inline
\newcommand\pythoninline[1]{{\pythonstyle\lstinline!#1!}}




%% Grafikpakete
\usepackage{graphicx}%


\usepackage{verbatim}

% How to diplay ESPResSo commands in flowing text. Larger code segments
% should be put inside boxes.
% \newcommand{\EScmd}[1]{\texttt{\textbf{#1}}}

\usepackage{listings}

\definecolor{codebg}{rgb}{0.95,0.95,0.95}
\definecolor{codeframe}{gray}{0.5}
\definecolor{codeshadow}{gray}{0.7}
\definecolor{codenumber}{rgb}{0.58,0,0.82}
\definecolor{pblau}{rgb}{0.09375,0.19921875,0.4296875}

\usepackage{listings}
\lstset{
  basicstyle=\ttfamily,
  keywordstyle=\bfseries\ttfamily\color[rgb]{0.65,0.16,0.18},
  identifierstyle=\ttfamily\color{pblau},
  commentstyle=\color[rgb]{0.133,0.545,0.133},
  stringstyle=\ttfamily\color[rgb]{0.627,0.126,0.941},
  showstringspaces=false,
  tabsize=2,
  breaklines=true,
  prebreak = \raisebox{0ex}[0ex][0ex]{\ensuremath{\hookleftarrow}},
  breakatwhitespace=false,
  numberstyle=\footnotesize \sf,
  numbers=none,
  stepnumber=1,
  numbersep=1.5em,
  xleftmargin=2.7em,
  framexleftmargin=2.7em,
  aboveskip={1\baselineskip},
  belowskip={1\baselineskip},
  columns=fixed,
  upquote=true,
  extendedchars=true,
  frame=shadowbox,
  rulesepcolor=\color{codeshadow},
  rulecolor=\color{codeframe},
  backgroundcolor=\color{codebg},
  mathescape=true
}


\begin{document}

\esptitlehead

\title{Setting up \es{}
\ifdefined\esversion%
\thanks{For \es \esversion}%
\fi%
}
\author{G. Rempfer and F. Weik}

\maketitle

\section{Introduction}
\es{} is distributed as source code, because this allows the user to disable features that are mutually exclusive or would otherwise decrease the performance unnecessarily. Users typically  build a custom \es{} binary for a specific simulation, containing only the minimal set of features required for this application. This tutorial will guide you through this whole process  from obtaining the source code of the development and the release version, to configuring and building binaries, to compiling the documentation.

\section{Getting the Source Code}
The procedure of building \es{} the development and the release versions differs only slightly.
\subsection{Development Version}
%
For the development version, first clone the \es{} reposity from github by running
\begin{lstlisting}[language=bash]
git clone https://github.com/espressomd/espresso.git
\end{lstlisting}
%
in a terminal. This creates a local copy of the current \es{} repository in the folder \verb!espresso!.
Enter the newly create directory using
\begin{lstlisting}[language=bash]
cd espresso
\end{lstlisting}
and run
\begin{lstlisting}[language=bash]
./bootstrap.sh
\end{lstlisting}
to configure the build system. This step is only necessary in the development version.

\subsection{Release Version}
The \es{} releases can be downloaded from the \es{} website at\\
\verb!http://espressomd.org/wordpress/download/!.
After downloading the latest version unpack it by running
\begin{lstlisting}[language=bash]
  tar xfv espresso-X.Y.Z.tar.gz
\end{lstlisting}
where X Y an Z have to be replaced with the actual version numbers.

\subsection{Compiling \es{}}
From here on the procedure is the same for both versions.
Change into the respective newly created \es{} directory and create another subfolder to hold your build and enter it by issuing
\begin{lstlisting}[language=bash]
mkdir build
cd build
\end{lstlisting}
You can maintain several such build folders with different feature sets using the same \es{} source folder. They don't have to reside within the source directory.
%
From within the build folder, run
%
\begin{lstlisting}[language=bash]
../configure
\end{lstlisting}
%
which create the makefiles. \es{} requires a number of external libraries, such as \texttt{libtcl} and others, depending on the feature set you plan to use. Most, if not all of these libraries will be available through your Linux distribution's repositories. If you manually install dependencies in a non-standard location, you have to specify this location during the \texttt{configure} stage of the build process. If, for example, you plan to use GPU features and your cuda toolkit is not located under \texttt{/usr/local/cuda}, you will have to specify its location using
%
\begin{lstlisting}[language=bash]
../configure --with-cuda=/PATH/TO/CUDA
\end{lstlisting}
%
A complete list of options can be obtained through
\begin{lstlisting}[language=bash]
../configure --help
\end{lstlisting}

Finally compile your version of \es{} with
%
\begin{lstlisting}[language=bash]
make -j 8
\end{lstlisting}
%
This produces an executable named \verb!Espresso! in the build directory which can be run by
\begin{lstlisting}[language=bash]
./Espresso
\end{lstlisting}

\subsection{User's guide}
You can build the \es{} manual called \emph{\es{} User's Guide} by running
\begin{lstlisting}[language=bash]
make ug
\end{lstlisting}
in the build directory. The User's Guide will be located under \texttt{build/doc/ug/ug.pdf}. It is an (almost) complete documentation covering all the features and commands available in \es{}. In addition, the User's Guide features some theoretical background information on some of the methods as well as references to the relevant literature.

\subsection{Configuring features}
%
After successfully running the configure script, your build folder will contain a file \texttt{myconfig-sample.hpp}. Create a copy of this file called \texttt{myconfig.hpp} by executing
%
\begin{lstlisting}[language=bash]
cp myconfig-sample.hpp myconfig.hpp
\end{lstlisting}
%
Then uncomment the respective lines in \texttt{myconfig.hpp} (by removing the leading slashes) to enable features your simulation requires. The User's Guide contains information about required features for every \es{} command.
When you activate or deactivate features, you have to \verb!rebuild! the binaries by running
%
\begin{lstlisting}[language=bash]
make -j 8
\end{lstlisting}
%
in the build directory.

\end{document}
